
\documentclass{article}

\usepackage{amssymb} %chekmark symbol

\begin{document}


%%%%%%%%%%%% PUBLICATIE UPLOADEN 
\subsubsection*{PUBLICATIE UPLOADEN VANUIT EIGEN FILESYSTEM}
\vspace{2 mm}

\textbf{ID}: FR-P001
\vspace{2 mm}



\hrule
\vspace{2 mm}
\noindent \textbf{Scenario}:
\begin{description}
\item CLIENT : G voegt bestand uit eigen filesystem toe, geeft bestand + titel + type (journal of proceeding), vertrekkend vanuit bepaalde bibliotheek
	\begin{description}
	\item \checkmark PDF file ? 
	\item $\rightarrow$ PUT: PDF + titel + type 
	\end{description}
	
\item SERVER : 
	\begin{description}
	\item \checkmark PDF file?
	\item \checkmark bestaat titel al? 
	\item $\leftarrow$ publication ID + status 
		\begin{description}
		\item status 200: publicatie bestond nog niet, continue 
		\item status 206: titel al aanwezig, request body bevat metadata publicatie  $\hookrightarrow$ A
		\end{description}
	\end{description}
	
\item CLIENT :  G wil metadata automatisch laten aanvullen
	\begin{description}
	\item $\rightarrow$ PUT:  titel  + type
	\end{description}
	
\item SERVER :  google scholar (GS) scraping
	\begin{description}
	\item $\leftarrow$ :    GS metadata voor type
	\end{description}
	
\item CLIENT :  G kan metadata manueel vervolledigen 
	\begin{description}
 	\item[titel * ]  \hfill 
  	\item[auteurs * ] \hfill 
  	\item[indien journal: naam/volume/nummer *] \hfill 
  	\item[indien proceeding: naam/organisatie *] \hfill 
  	\item[jaar van publicatie * ] \hfill 
  	\item[onderzoeksdomeinen *]  \hfill 
  	\item[keywords * ]  \hfill 
  	\item[abstract] \hfill 
   	\item[aantal citaties] \hfill 
    	\item[URL] \hfill 
  	\end{description}
	
	\begin{description}
	\item \checkmark validatie
	\item $\rightarrow$ PUT:  metadata
	\end{description}
	
\item SERVER :  
	\begin{description}
	\item \checkmark validatie
	\item \checkmark zijn auteurs al bekend in systeem? (dit gaat niet enkel over gebruikers = auteurs die ook een account hebben)
	\item $\leftarrow$ :   status
			\begin{description}
			\item status 200: auteur nog niet bekend, continue 
			\item status 206: auteur al aanwezig, request body bevat JSON met de titels van maximaal vijf publicaties van de auteur ( ! per gevonden auteur, kunnen er meerdere zijn )   $\hookrightarrow$ B
			\end{description}
	\end{description}
	
\item CLIENT :  message: publicatie toegevoegd aan huidige bibliotheek.

 \end{description}
 
 
\vspace{2 mm}
\hrule
\vspace{2 mm}
 
\noindent Alternatieve scenario's: 

\begin{description}

\item A: de titel van de publicatie komt al voor in ons systeem. 
 	\begin{description}
 	\item CLIENT : haalt metadata van publicatie uit request body en geeft deze weer aan gebruiker
 		\begin{description}
		\item vraag aan gebruiker:  \emph{'is this the publication you are looking for'?}. 
		\item indien ja:  $\rightarrow$ PUT:  ...? 
		\item indien nee: continue
		\end{description}
  	\end{description}
\item B:  auteur(s) zijn al bekend in ons systeem
 	\begin{description}
 	\item CLIENT : haalt per gevonden auteur een lijst met publicatie titels uit request body en geeft weer. Gebruiker kan aan de hand van titels oordelen of het over de correcte auteur gaat. 
 		\begin{description}
		\item vraag aan gebruiker:  \emph{'is this the author you are looking for'?}. 
		\item indien ja:  $\rightarrow$ PUT:  ...? 
		\item indien nee: continue
		\end{description}
  	\end{description}
  
 \end{description}



\end{document}





